% Created 2014-04-23 Wed 23:42
\documentclass[CJK]{cctart}
\usepackage[T1]{fontenc}
\usepackage{fixltx2e}
\usepackage{graphicx}
\usepackage{longtable}
\usepackage{float}
\usepackage{wrapfig}
\usepackage{soul}
\usepackage{textcomp}
\usepackage{marvosym}
\usepackage{wasysym}
\usepackage{latexsym}
\usepackage{amssymb}
\usepackage{hyperref}
\tolerance=1000
\providecommand{\alert}[1]{\textbf{#1}}

\title{Work}
\author{Yalong Zhao}
\date{\today}
\hypersetup{
  pdfkeywords={},
  pdfsubject={},
  pdfcreator={Emacs Org-mode version 7.9.3f}}

\begin{document}

\maketitle

\setcounter{tocdepth}{3}
\tableofcontents
\vspace*{1cm}

\section{Internship in Panasonic, Singapore}
\label{sec-1}
\subsection{NEC implement}
\label{sec-1-1}
\subsubsection{Introduction}
\label{sec-1-1-1}

NEC implement project is based on the paper ``\href{http://arxiv.org/pdf/1212.6094v1.pdf}{Large Scale Strongly Supervised Ensemble Metric Learning, with Applications to Face Verification and Retrieval}'' wrote by Chang Huang, Shenghuo Zhu and Kai Yu NEC
laboratories. This NEC paper is about one success frame for face
recognition.
\subsubsection{Goal}
\label{sec-1-1-2}

\begin{itemize}
\item Run the Matlab code write by Zhen Wei.
\item Write the C++ version of NEC paper.
\item Implement the CMD feature descriptor.
\item Understand why this framework performance good on face recognition.
\item Integrate the good feature of NEC framework into company method.
\end{itemize}
\subsubsection{Completed}
\label{sec-1-1-3}

\begin{enumerate}
\item Run some test based on Matlab code, get some results.

   For the origin setting in paper, they followed the ``unrestricted
   configuration'' that allows authors to use the identity information of
   training data and choose the ``aligned'' version of LFW\footnote{Labeled faces in the Wild, the link is
\end{enumerate}
\href{http://vis-www.cs.umass.edu/lfw/}{http://vis-www.cs.umass.edu/lfw/}.
 }. The
   database contains 10 folds, every folds contains 600 pairs face
   images, 9 folds for training and remain 1 fold for testing.
   
   In the paper's experiments, 13260 rectangular regions of size
   varying from \texttt{8*8} to \texttt{96*144} are enumerated within the \texttt{110*150}
   region. Each rectangular region defines a covariance matrix
   descriptor and a SLBP descriptor; every descriptor is whitened on
   training data by PCA; top 45 dimensions of SLBP descriptors are
   preserved. Eventually, each face image is represented by 13260
   covariance matrix descriptors and 13260 SLBP descriptors, each of
   45 dimensions constituting a feature group.
   
   \textbf{But for my experiments, because the program is too slow, I changed the 13260 blocks to 678, which contain the important    position of face: eyes, nose, mouse and big face rectangular.}

   \emph{Results of experiments:}

\begin{center}
\begin{tabular}{rrrrll}
 No.  &  Train Num  &   mu  &  Block Num  &  Accuracy  &  Cost Time     \\
\hline
   1  &         20  &  100  &        687  &  61.50\%   &  2 hours       \\
   2  &         20  &  200  &        687  &  64.50\%   &  6 hours       \\
   3  &         20  &  300  &        687  &  65.00\%   &  12 hours      \\
   4  &         50  &  100  &        687  &  64.50\%   &  8 hours       \\
   5  &         50  &  200  &        687  &  65.50\%   &  26 hours      \\
   6  &        100  &  200  &        687  &  68.00\%   &  About 1 week  \\
\end{tabular}
\end{center}



\begin{enumerate}
\item Implement the CMD descriptor in Matlab version, but seems
   something error.
\item Rewrite the algorithm use C++.
\item Test the GPU, OpenMP, Eigen and OpenCV for solving the eigen value problem.
\end{enumerate}
\subsubsection{Schedule}
\label{sec-1-1-4}
\begin{itemize}

\item \textbf{CANCELLED} Solve the problem of CMD descriptor. \textbf{:NEC:CANCELLED:}\\
\label{sec-1-1-4-1}%
\texttt{DEADLINE:} \textit{2014-02-28 Fri} \texttt{SCHEDULED:} \textit{2014-02-07 Fri}

The Matlab version CMD descriptor has some problems, more train
images leads to low accuracy. I need find out the result.

Why cancelled: I have to back to China, the work of internship is finished.


\item \textbf{DONE} Write C++ version of CMD descriptor. \textbf{:NEC:}\\
\label{sec-1-1-4-2}%
\texttt{DEADLINE:} \textit{2014-02-11 Tue} \texttt{SCHEDULED:} \textit{2014-02-07 Fri}

For the C++ version of NEC implement, it just contains the ULBP
feature, I need add the CMD feature.


\item \textbf{DONE} Find the error of why C++ version ULBP is so slow? \textbf{:NEC:}\\
\label{sec-1-1-4-3}%
\texttt{DEADLINE:} \textit{2014-02-28 Fri} \texttt{SCHEDULED:} \textit{2014-02-10 Mon}

The C++ version of NEC implement used ULBP feature runs very slow,
about update to the 30 blocks the program begin very very slow.

Result: It is because the eigen decomposition is too slow in Eigen,
so after 100 , the matrix is 4500*4500, need very long time.


\item \textbf{CANCELLED} Test the MAGMA CUDA library. \textbf{:NEC:CANCELLED:}\\
\label{sec-1-1-4-4}%
\texttt{DEADLINE:} \textit{2014-02-17 Mon} \texttt{SCHEDULED:} \textit{2014-02-12 Wed}

If I want to use CUDA for speeding up the C++ version program, I need
to find a C++ CUDA matrix computation library, especially for solving
the eigen value problem.

Why cancelled: I have to back to China, the work of internship is finished.


\item \textbf{DONE} Add the OpenMP in C++ NEC implement. \textbf{:NEC:}\\
\label{sec-1-1-4-5}%
\texttt{DEADLINE:} \textit{2014-02-28 Fri} \texttt{SCHEDULED:} \textit{2014-02-10 Mon}

According my test, the OpenMp is useful for speeding up the program,
but for calculating the eigen value, how to use OpenMP to speed up?

\end{itemize} % ends low level
\subsection{Face Point Localization}
\label{sec-1-2}

Face Point Localization project is main about use web camera to
detect the landmarks on face, then use the information of eye gaze to
utilize some applications, such as zoom in or out, video play and
games.
\subsection{Face Identification}
\label{sec-1-3}
\subsubsection{Introduction}
\label{sec-1-3-1}

Face Identification project aim to label the name of actor/actress on
the scene of one video. This project is cooperate with NUS Prof.Yan's
student @Luoqi and @Junshi.

For me and now, I need to help them find more data for training.
\subsubsection{Goal}
\label{sec-1-3-2}
\label{ecbc34be-f033-4ce4-9b44-6f63c2a441a5}

\begin{itemize}
\item Label the actor/actress's name in high accuracy.
\item Find more videos from YouTube.
\item Find more DVDs which is high quality data from Rokutan and Amazon.
\item Help them to label the face.
\item Received Junshi's face identification source code.\textit{2014-02-17 Mon}
\item I need do the rest train and test myself.
\end{itemize}
\subsubsection{Completed}
\label{sec-1-3-3}

Find 281 videos from YouTube and helped them to label the face position.
\subsubsection{Schedule}
\label{sec-1-3-4}
\begin{itemize}

\item \textbf{DONE} Now is searching the DVDs. \textbf{:Face\_Identification:}\\
\label{sec-1-3-4-1}%
\texttt{DEADLINE:} \textit{2014-02-21 Fri}


\item \textbf{DONE} Ask Shen about DVDs on TaoBao. \textbf{:Face\_Identification:}
\label{sec-1-3-4-2}%

\item \textbf{DONE} Buy the first 3 person's DVDs to test if these videos can be used. \textbf{:Face\_Identification:}\\
\label{sec-1-3-4-3}%
\texttt{DEADLINE:} \textit{2014-02-07 Fri} \texttt{SCHEDULED:} \textit{2014-02-07 Fri}


\item \textbf{DONE} Make sure if I need to buy the DVDs from Nex in Searengoon. \textbf{:Face\_Identification:}\\
\label{sec-1-3-4-4}%
\texttt{DEADLINE:} \textit{2014-02-07 Fri} \texttt{SCHEDULED:} \textit{2014-02-07 Fri}


\item \textbf{DONE} Go to Nex to buy the Jap DVDs. \textbf{:Face\_Identification:}\\
\label{sec-1-3-4-5}%
\texttt{DEADLINE:} \textit{2014-02-09 Sun} \texttt{SCHEDULED:} \textit{2014-02-08 Sat}


\item \textbf{DONE} Print the actor name list both in Jp and En. \textbf{:Face\_Identification:}\\
\label{sec-1-3-4-6}%
\texttt{DEADLINE:} \textit{2014-02-07 Fri} \texttt{SCHEDULED:} \textit{2014-02-07 Fri}


\item \textbf{DONE} Write the Payment Voucher for buying the DVDs from Nex.\\
\label{sec-1-3-4-7}%
\texttt{SCHEDULED:} \textit{2014-02-10 Mon}


\item \textbf{DONE} Test Junshi's code. Add on \textit{2014-02-17 Mon} \textbf{:Face\_Identification:}
\label{sec-1-3-4-8}%

\item \textbf{DONE} Build the dataset based on DVD. Add on \textit{2014-02-17 Mon}\textbf{:Face\_Identification:}
\label{sec-1-3-4-9}%

\item \textbf{DONE} Train the new mode based on new database. Add on \textit{2014-02-17 Mon} \textbf{:Face\_Identification:}
\label{sec-1-3-4-10}%
\end{itemize} % ends low level
\subsection{Work trivia}
\label{sec-1-4}
\subsubsection{\textbf{DONE} Fill the PRDCSG Target Plan and Skill Up Plan System.}
\label{sec-1-4-1}
\label{9cc1900b-38f1-4170-9f4d-5d7fe7e1e7da}

    \texttt{DEADLINE:} \textit{2014-02-19 Wed} \texttt{SCHEDULED:} \textit{2014-02-12 Wed}
\subsubsection{\textbf{DONE} Write the year report slides.}
\label{sec-1-4-2}
\label{30b49261-ad0c-4573-af40-5b3d52844df0}

    \texttt{DEADLINE:} \textit{2014-02-12 Wed} \texttt{SCHEDULED:} \textit{2014-02-11 Tue}
\subsubsection{\textbf{DONE} ISG talk.}
\label{sec-1-4-3}
\label{b16d4f66-6ea8-46dc-b2bc-5e7f7652e49d}

    \texttt{SCHEDULED:} \textit{2014-03-07 Fri}
\subsubsection{\textbf{DONE} Label data for JK.}
\label{sec-1-4-4}
\label{1bbcfacc-6103-4979-893e-ada40e295d3b}

    \texttt{SCHEDULED:} \textit{2014-02-26 Wed}
\subsubsection{\textbf{DONE} Install Ubuntu on new hard disk, and test CNN and CUDA environment.}
\label{sec-1-4-5}
\label{d361d334-f20c-49d8-bf50-84f8ebd99bad}
\section{Master graduate thsis}
\label{sec-2}
\subsection{Master-Degree Proposal}
\label{sec-2-1}

\end{document}
